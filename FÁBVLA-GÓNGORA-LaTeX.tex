\documentclass[11pt,a4paper,twoside]{article}
\usepackage[noeledsec,noend,noledgroup,nopenalties,series={A,E},parapparatus]{reledmac}
\usepackage{fontspec}
\usepackage[paperwidth=17.50cm,paperheight=25.50cm,top=1.40cm,inner=1.95cm,outer=1.65cm,bottom=2.50cm]{geometry}
\usepackage{polyglossia}
\usepackage[breaklinks=true,hidelinks]{hyperref}
\usepackage[backend=biber,style=apa,url=true]{biblatex}
\addbibresource{bibliography.bib}
\title{FÁBVLA DE POLIFEMO Y GALATEA}
\author{\fontsize{14}{11.96}\selectfont LVIS DE GÓNGORA}
\date{\vspace{60pt}\fontsize{11}{11}\selectfont{TEXTO Y VARIANTES DE LA REDACCIÓN PRIMITIVA}\\\fontsize{40}{11}\selectfont{\vfill\LaTeX}}
%
\setmainfont{CommonSerif}[BoldFont=,Numbers=Lining]
\setdefaultlanguage{spanish}
%
\setlength{\parindent}{0.6cm}
\setlength{\bibhang}{0.6cm}
%
\fnpos{critical-familiar}
\fnpos{{A}{familiar},{A}{critical},{E}{familiar}}
\Xarrangement{paragraph}
\parindentX
\AtBeginDocument{\Xmaxhnotes{0.66\textheight}\maxhnotesX{1\textheight}}
\Xafterlemmaseparator{0.4em}
\Xbeforelemmaseparator{-0.17em}
\Xafternumber{0.4em}
\Xnumberonlyfirstinline
\Xnonbreakableafternumber
\Xinplaceoflemmaseparator{0em}
\newcommand{\comillas}[1]{«#1»}
%
\prenotesX{11pt}
\afterruleX{2pt}
\Xprenotes{11pt}
\Xafterrule{2pt}
%
%
\usepackage{tocloft}
\renewcommand{\cftsecleader}{\cftdotfill{\cftdotsep}}
%Espaciado de la lista.
\usepackage{enumitem}
\setlist{nosep,noitemsep,parsep=0pt}
%
%
\begin{document}
	{\pagenumbering{gobble}
			\maketitle
		}
		\newpage%

\hskip0pt
\vfill
\begin{flushright}
    Edición preparada por {\fontsize{9}{1}\selectfont I. BRACHAMONS LETESMVS}.
\end{flushright}

\begin{flushright}
	Documento compuesto en LuaLaTeX con el paquete Reledmac de {\fontsize{9}{11}\selectfont MAÏAEVL ROVQVETTE}.
\end{flushright}
\newpage
%
\tableofcontents
\newpage
%
\pagenumbering{arabic}
\setcounter{page}{1}
\section*{\fontsize{11}{14}\selectfont TESTIMONIOS CONSVLTADOS}
\addcontentsline{toc}{section}{Testimonios consultados}

\textit{Ch} = \textit{\fontsize{9}{11}\selectfont OBRAS DE D. LVIS DE GONGORA} [BNE, RES/45].
%
Manuscrito en papel con los textos de Góngora; 26x18cm.; 3 vols.; copiado por Antonio Chacón y supervisadas por el autor, según así lo expresa Chacón en la portada con el texto «Reconocidas i comunicadas con el {\fontsize{9}{11}\selectfont POR D. ANTONIO CHACON PONCE} de Leon»; dedicado a don Gaspar de Guzmán, con la rúbrica «{\fontsize{9}{11}\selectfont AL EXC.MO SEÑOR D. GASPAR DE GVZMAN CONDE DE OLIVARES, DVQVE DE SANLVCAR}», fechado en «diziembre 12 de [1]628»; las pp. 121-137 del vol. I contienen el \textit{Polifemo}. <\url{https://bdh-rd.bne.es/viewer.vm?id=0000015414&page=1}>

\textit{LVi} = \textit{\fontsize{9}{11}\selectfont \textit{OBRAS EN VERSO DEL HOMERO ESPAÑOL}} \textit{que recogio Iuan Lopez de Vicuña}.
%
Edición impresa; editio princeps de las obras poéticas de Góngora (?); fecha de impresión 1627 en portada, «Suma de Tassa», fe de erratas y prólogo, y en 1620 fechada la aprobación del P. fray Juan Gómez y del Maestro Vicente Espinel; «recogidas» por el editor desde 1607 hasta 1620 y publicadas sin autorización del autor \parencite{Alonso1963}, causa que llevó a López de Vicuña a declarar ante la Inquisición, quien luego admitió haber recibido un manuscrito con la colección de textos gongorinos de «don Juan de Salierne»\footnoteE{\comillas{[...] dixo que abía de siete a ocho años, que un don Juan de Salierne, vezino desta villa, ya difunto, tenía recojidas todas las obras de don Luis de Góngora en un libro manuscripto y trató de imprimillas, para lo qual se sacó privilegio en caveza deste que declara, que era muy su amigo, y además le dio por él trecientos y cinquenta reales y se le entregó a este testigo con las censuras y recaudos nezesarios, y por entonzes no trató de la impresión por aver entendido que el dicho don Luis de Góngora no gustava de que en su vida se imprimiese y habrá como cinco o seis meses, que haviendo muerto el dicho don Luis, trató este que declara con Alonso Pérez, mercader de libros, de que se imprimiese el dicho libro, como se hizo en virtud del dicho privilegio [...]}. Expediente inquisitorial transcrito por \textcite{Moll1997}.}; el \comillas{manuscrito Salierne} porta un testimonio adecuado \parencite{Alonso1963}; los ff. 113\textit{v.}-121\textit{v.} contienen el \textit{Polifemo}.

\textit{G1} = \textit{Obras de Luis de Góngora} [BNE, MSS/22217].
%
Manuscrito en papel con las obras de Góngora; 248 fols.; 21x15cm.; entre 1601 y 1700?; pertenece a la redacción primitiva del texto, que Góngora modificó --probablemente-- por influencia de su amigo Pedro de Valencia, quien criticó su obra en una carta fechada el «30 de junio 1613»\footnoteE{Se conservaron dos redacciones autógrafas de esta carta; una, la más antigua, en el MSS/5585 [165\textsuperscript{r.}-168\textsuperscript{v.}]; la segunda \parencite{PérezLópez1988}, en el MSS/3906 [ff. 64\textsuperscript{r.}-67\textsuperscript{r.}], que tiene su copia en el MSS/19004 [ff. 13\textsuperscript{r.}-20\textsuperscript{r.}], que apenas modifica las grafías, como la conjunción copulativa escrita con \textit{i} en el autógrafo, cuya copia redacta con \textit{y}. Todos los manuscritos yacen guardados en la Biblioteca Nacional de España. Allí, en esa carta, redacta Pedro de Valencia su crítica a las recientes \textit{Soledades} y \textit{Fábula de Polifemo y Galatea}.}; los ff. 2\textsuperscript{r.}-16\textsuperscript{v.} contienen el \textit{Polifemo}. <\url{https://bdh-rd.bne.es/viewer.vm?id=0000012206&page=1}>

\textit{BNE*} = \textit{Obras} [BNE, RES/45]. 
%
Manuscrito en papel; 363 págs.; 21x15cm.; entre 1601 y 1700?; contiene noticias biográficas de Luis de Góngora y Leonardo de Argensola y producciones de ambos; las pp. 33-52 contienen el \textit{Polifemo}. <\url{https://bdh-rd.bne.es/viewer.vm?id=0000141426&page=1}>

\section*{\fontsize{11}{14}\selectfont CRITERIOS DE EDICIÓN}
\addcontentsline{toc}{section}{Criterios de edición}

Transcripción del Códice Chacón, porque contiene la redacción definitiva del texto del \textit{Polifemo}. Cuando \textit{Ch} lee erróneamente o sus lecciones parecen ser contrarias al autor, se adoptarán las lecciones de \textit{LVi} y, si este erra, se adoptarán las de Pellicer. Es necesario destacar que \textit{Ch} tiene la supervisión del autor y, por tanto, posee carácter de idiógrafo. 

Criterios de presentación ortográficos: 
%
\begin{itemize}[label=--]%
\item \textit{v}-\textit{u} = se quita la alternancia y se designa \textit{v} para consonantes y \textit{u} para vocales.
\item \textit{tt}-\textit{rr}-\textit{ss}-\textit{ff} [excepto \textit{ll} y otros dobles en nombres] = se simplifican.
\item \textit{i}-\textit{j}-\textit{y} = se regulariza la alternancia. Se coloca \textit{i} para vocales, \textit{j }para consonantes e \textit{y} para la conjunción copulativa o vocablos finalizados en \textit{y} vocal.
\item \textit{h} = se añade donde no las hay, por ejemplo, «oy» [hoy].
\item Los excesivos leísmos de \textit{Ch} se regularizan.
\item \textit{ch} debe leerse como \textit{k}-\textit{kh} en vocablos provenientes del griego --pensados a través del latín--, tal como \textit{Baccho} [Baco], \textit{echo} [eco], \textit{choro} [coro], etc.; y en el resto de vocablos, por ejemplo, en \textit{corcho}, \textit{ch} debe leerse con los criterios fonéticos actuales.
\item Se adopta la puntuación propuesta por la edición de \textcite{Alonso1967}.
\item Se sigue la acentuación moderna.%
\end{itemize}
%

El \comillas{aparato crítico} es negativo. Se reproducen los errores de \textit{Ch} y todas las variantes de \textit{G1} en el \comillas{aparato crítico}. Los \textit{lemmata} se colocaron tal cual se editaron para el texto crítico, porque estos servirán solo de referencia para el análisis posterior de la evolución del texto.

\section*{\fontsize{11}{14}\selectfont BIBLIOGRAFÍA CONSVLTADA}
\addcontentsline{toc}{section}{Bibliografía consultada}
\nocite{*}
\printbibliography[heading=none]
\newpage

\section*{\centering \fontsize{11}{14}\selectfont[DEDICATORIA]}
\addcontentsline{toc}{section}{Dedicatoria}
\vspace{-1em}
\begin{center}
	[I]
\end{center}
\beginnumbering
\pstart
{\fontsize{9}{11}\selectfont ESTAS QUE ME DICTÓ}, rimas sonoras,\\
culta si, aunque bucólica, Thalía\\
--¡oh excelso conde!--, en las purpúreas horas\\
que es rosas la Alva, y rosicler el día,\\
ahora que de luz tu Niebla doras,\\
escucha, al son de la çampoña mía,\\
si ya los muros no te ven, de Huelva,\\
peinar el \edtext{viento}{\Afootnote{monte \textit{corr. G1}}}, fatigar la selva.\par\pend
\begin{center}
	[II]
\end{center}
\pstart
Templado, pula en la maestra mano\\
el generoso páxaro su pluma,\\
o tan mudo en la alcándara, que en vano\\
aun desmentir al cascabel presuma;\\
tascando haga el freno de oro, cano,\\
del cavallo Andaluz la ociosa espuma;\\
gima el lebrel en el cordón de seda.\\
Y al cuerno, al fin, la cíthara suceda.\par\pend
\begin{center}
	[III]
\end{center}
\pstart
Treguas al exercicio sean robusto,\\
ocio atento, silencio dulce, en quanto\\
debaxo escuchas de dosel augusto,\\
del músico jayán, el fiero canto.\\
Alterna con las Musas hoy el gusto;\\
que si la mía puede ofrecer tanto,\\
clarín (y de la Fama no segundo),\\
tu nombre oirán los términos del mundo.\par\pend

\section*{\centering \fontsize{11}{14}\selectfont[FÁBVLA]}
\addcontentsline{toc}{section}{Fábula}
\vspace{-1em}
\begin{center}
	[IV]
\end{center}
\pstart
Donde espumoso el mar sicilïano\\
el pie argenta de plata al Lilybeo\\
(bóbeda o de las fraguas de Vulcano,\\
o tumba de los huesos de Tipheo),\\
pállidas señas ceniçoso un llano\\
--quando no del sacrílego deseo--,\\
del duro oficio da. Allí una alta roca\\
mordaça es a una gruta de su boca.\par\pend

\begin{center}
	[V]
\end{center}
\pstart
Guarnición tosca de este escollo duro\\
troncos robustos son, a cuya greña\\
menos luz deve, menos aire puro\\
la caverna profunda, que a la peña;\\
caliginoso lecho, el seno obscuro\\
ser de la negra noche nos lo enseña\\
infame turba de nocturnas aves,\\
gimiendo tristes y bolando graves.\par\pend

\begin{center}
	[VI]
\end{center}\pstart
De este, pues, formidable de la tierra\\
bosteço, el melancólico vazío\\
\edtext{a Poliphemo, horror de aquella sierra}{\Afootnote{al cabrero mayor de aquella sierra \textit{G1}}},\\
bárbara choça es, alvergue umbrío\\
y redil espacioso, donde encierra\\
quanto las cumbres ásperas cabrío,\\
de los montes, esconde: copia bella,\\
que un silvo junta y un peñasco sella.\par\pend

\begin{center}
	[VII]
\end{center}\pstart
Un monte era de miembros eminente\\
este (que, de Neptuno hijo fiero,\\
\edtext{de un ojo illustra el orbe de su frente}{\Afootnote{el medio orbe illustraua de su frente \textit{G1}}},\\
\edtext{émulo casi del}{\Afootnote{un ojo del \textit{G1}}} mayor luzero)\\
cíclope, a quien el pino más valiente,\\
bastón, lo obedecía, tan ligero,\\
y al grave peso junco tan delgado,\\
que un día era bastón y otro cayado.\par\pend

\begin{center}
	[VIII]
\end{center}\pstart
Negro el cabello, imitador undoso\\
de las obscuras aguas del Leteo,\\
al viento que lo peina proceloso\\
vuela sin orden, pende sin aseo;\\
un torrente es su barba impetüoso,\\
que (adusto hijo de este Pirineo)\\
su pecho inunda, o tarde, o mal, o en vano\\
surcada aun de los dedos de su mano.\par\pend

\begin{center}
	[IX]
\end{center}\pstart
No a la Trinacria en sus montañas, fiera\\
armó de crüeldad, calzó de viento,\\
que \edtext{redima}{\Afootnote{redimia \textit{G1}}} feroz, salve ligera,\\
su piel manchada de colores ciento:\\
pellico es ya la que en los \edtext{bosques}{\Afootnote{montes \textit{G1}}} era\\
mortal horror al que con paso lento\\
los bueyes a su alvergue reducía,\\
pisando la dudosa luz del día.\par\pend

\begin{center}
	[X]
\end{center}\pstart
Cercado es (quanto más capaz, mas lleno)\\
de la fruta el zurrón, casi abortada,\\
que el tardo otoño dexa al blando seno\\
de la piadosa hierba, encomendada:\\
\edtext{la serva, a quien leda rugas el heno}{\Afootnote{la delicada selua aquien el heno \textit{G1}}};\\
\edtext{la pera, de quien fue cuna dorada}{\Afootnote{rugas la da en la cuna la opilada \textit{G1}}}\\
\edtext{la rubia paja; y --pálida tutora--}{\Afootnote{camuesa, del color pierde amarillo \textit{G1}}}\\
\edtext{la niega avara, y prodiga la dora}{\Afootnote{en tomando el azero del cuchillo \textit{G1}}}\footnoteE{77--80 = en el f. 65\textsuperscript{v.} del MS/3096 --el autógrafo de la segunda redacción de la carta--, Pedro de Valencia enuncia, respecto a los pasajes, contrastándolos con las \textit{Soledades}: «[...] i como en casi todo el discurso destas Soledades, alta i grandiosame\textit{n}te con sencilleza i claridad, co\textit{n} breves periodos i los vocablos en sus lugares, i no se vaya co\textit{n} pretension de grandeza i altura a buscar i imitar lo estraño oscuro, ageno, i no tal como lo q\textit{ue} a v. m. le nasce en casa; i no me diga que la camuesa pierde el color amarillo, en tomando el azero del cuchillo [...]». En el apunte sexto, antes de presentar estas variantes, Pellicer en su \textit{Lecciones solemnes} enuncia: «De la tutela harto los I. C. en algunos M. S. se lee la mitad desta estancia distintamente, y no se si diga mejor, […] En \textit{modandola}, aludiendo a la enfermedad de la opilación, contraida de comer barro, y de la mucha agua, tan frecuente en las damas de España: para cuyo remedio es vtil la \textit{flor de azero}, o \textit{la escama}, y el andar, como siente \textit{Galeno li. 9. Simpl.} y \textit{Dioscorides lib. 5}, \textit{c. 49.}». La única variante de Pellicer respecto a \textit{G1} está en \textit{serua}, variante solo gráfica.}.\par\pend

\begin{center}
	[XI]
\end{center}\pstart
Erizo es el zurrón, de la castaña,\\
y (entre el membrillo o verde o datilado)\\
de la manzana hipócrita, que engaña\\
a lo pálido no, a lo arrebolado,\\
y, de la encina (honor de la montaña,\\
que pavellón al siglo fue dorado)\\
el tributo, alimento, aunque grosero,\\
del \edtext{mejor}{\Afootnote{primer \textit{G1}}} mundo, del candor primero.\par\pend
\relax
\vfill
\newpage

\nopagebreak
\begin{center}
	[XII]
\end{center}\pstart
Cera y cáñamo unió (que no debiera)\\
cient cañas, cuyo bárbaro rüido,\\
de más echos que unió cáñamo y cera\\
albogues, duramente es repetido.\\
La selva se confunde, el mar se altera,\\
rompe Tritón su caracol torcido,\\
sordo huye el baxel a vela y remo:\\
¡tal la música es de Poliphemo!\par\pend

\begin{center}
	[XIII]
\end{center}\pstart
Nimpha, de Doris hija, la más bella\\
adora, que vio el reino de la espuma.\\
Galathea es su nombre, y dulce en ella\\
el terno Venus de sus gracias summa.\\
Son una y otra luminosa estrella\\
lucientes ojos de su blanca pluma:\\
si roca de cristal no es de Neptuno,\\
pavón de Venus es, cisne de Juno.\par\pend 

\begin{center}
	[XIV]
\end{center}\pstart
Purpureas \edtext{rosas}{\Afootnote{ojas \textit{G1}}} sobre Galathea\\
la Alva entre lilios cándidos deshoja:\\
duda el Amor quál más su color sea,\\
o púrpura nevada, o nieve roja.\\
De su frente la perla es, erithrea,\\
émula vana. El ciego dios se enoja,\\
y, \edtext{condenado}{\Afootnote{condenando \textit{G1}}} su esplendor, la deja\\
pender en oro al nácar de su oreja.\par\pend

\begin{center}
	[XV]
\end{center}\pstart
Invidia de las nimphas y cuidado\\
de quantas \edtext{honra el mar}{\Afootnote{honra el amor \textit{G1}}} deidades era;\\
pompa del marinero niño alado\\
que sin fanal conduce su venera.\\
Verde el cabello, el pecho no escamado,\\
ronco sí, escucha a Glauco la ribera\\
inducir a pisar la bella ingrata,\\
en carro de cristal, campos de plata.\par\pend
\relax
\vfill
\newpage

\begin{center}
	[XVI]
\end{center}\pstart
Marino joven, las cerúleas sienes,\\
del más tierno coral ciñe Palemo,\\
rico de quantos la agua engendra bienes\\
del Pharo odioso al promontorio extremo;\\
mas en la gracia igual, si en los desdenes\\
perdonado algo más que Poliphemo,\\
de la que, aún no le oyó, y, calzaba plumas,\\
tantas flores pisó como él espumas.\par\pend

\begin{center}
	[XVII]
\end{center}\pstart
Huye la nimpha bella; y el marino\\
amante nadador, ser bien quisiera,\\
ya que no áspid a su pie divino,\\
dorado pomo a su veloz carrera;\\
mas, ¿quál diente mortal, quál metal fino\\
la fuga suspender podrá ligera,\\
que el desdén solicita? ¡Oh quánto yerra\\
delphín, que sigue en agua corza en tierra!\par\pend

\begin{center}
	[XVIII]
\end{center}\pstart
Sicilia, en quanto oculta, en quanto ofrece,\\
copa es de Baccho, huerto de Pomona:\\
tanto de frutas ésta la enriquece,\\
quanto aquel de racimos la corona.\\
En carro que estival trillo parece,\\
a sus compañas Ceres no perdona,\\
de cuyas siempre fértiles espigas\\
las provincias de Europa son hormigas.\par\pend

\begin{center}
	[XIX]
\end{center}\pstart
A \edtext{Pales}{\Afootnote{Palas \textit{G1}}}\footnoteE{Pales, divinidad del ganado. \textit{G1} lee «Palas» por Palas Atenea, que pudo ser un error del copista al leer --probablemente-- \textit{a} por \textit{e}, o, quizá, es una lección que perteneció enteramente al texto primitivo.} su viciosa cumbre deve\\
lo que a Ceres, y aún más, su vega llana;\\
pues si en la una granos de oro llueve,\\
copos nieva en la otra mill de lana.\\
De quantos siegan oro, esquilan nieve,\\
o en pipas guardan la exprimida grana,\\
bien sea religión, bien amor sea,\\
deidad, aunque sin templo, es Galathea.\par\pend

\begin{center}
	[XX]
\end{center}\pstart
Sin aras, no: que el margen donde para\\
del espumoso mar su pie ligero,\\
al labrador de sus primicias ara,\\
\edtext{de sus esquilmos es al ganadero}{\Afootnote{y sus squilmos es de ganadero \textit{G1}}};\\
de la Copia --a la tierra poco avara--\\
el cuerno vierte el hortelano, entero,\\
sobre la mimbre que texió, prolixa,\\
si artificiosa no, su honesta hija.\par\pend

\begin{center}
	[XXI]
\end{center}\pstart
Arde la juventud, y los arados\\
peinan las tierras que surcaron antes,\\
mal conducidos, quando no arrastrados\\
de tardos bueyes, qual su dueño errantes;\\
sin pastor que los silve, los ganados\\
los cruxidos ignoran resonantes,\\
de las hondas, si, en vez del pastor pobre,\\
el zéphiro no silva, o cruxe el robre.\par\pend

\begin{center}
	[XXII]
\end{center}\pstart
Mudo la noche el can, el día, dormido,\\
de cerro en cerro y sombra en sombra yace.\\
Bala el ganado; al mísero valido,\\
nocturno el lobo de las sombras nace.\\
Cévase; y fiero, dexa humedecido\\
en sangre de una lo que la otra pace.\\
¡Revoca, Amor, los silvos, o a su dueño\\
\edtext{el silencio del can siga, y el sueño}{\Afootnote{el silençio del can sigan o el sueño \textit{G1}}}!\par\pend

\begin{center}
	[XXIII]
\end{center}\pstart
La fugitiva nimpha, en tanto, donde\\
hurta un laurel su tronco al sol ardiente,\\
tantos jazmines quanta hierba esconde\\
la nieve de sus miembros, da a una fuente.\\
Dulce se quexa, dulce le responde\\
un ruiseñor a otro, y dulcemente\\
al sueño da sus ojos la armonía,\\
por no abrasar con tres soles al día.\par\pend
\relax
\vfill
\newpage

\begin{center}
	[XXIV]
\end{center}\pstart
Salamandria del sol, vestido estrellas,\\
latiendo el Can del cielo estava, quando\\
(polvo el cabello, húmidas centellas,\\
si no ardientes aljófares, sudando)\\
llegó Acis; y de ambas luzes bellas\\
dulce Occidente viendo al sueño blando,\\
su boca dio, y sus ojos quanto pudo,\\
al sonoro cristal, al cristal mudo.\par\pend

\begin{center}
	[XXV]
\end{center}\pstart
Era Acis un benablo de Cupido,\\
de un fauno, medio hombre, medio fiera,\\
en Simetis, hermosa nimpha, avido;\\
gloria del mar, honor de su ribera.\\
El bello imán, el ídolo dormido,\\
que acero sigue, idólatra venera,\\
rico de quanto el huerto ofrece pobre,\\
rinden las bacas y fomenta el robre.\par\pend

\begin{center}
	[XXVI]
\end{center}\pstart
El celestial humor recién quaxado,\\
que la almendra guardó entre verde y seca,\\
en blanca mimbre se lo puso al lado,\\
y un copo en verdes juncos, de manteca;\\
en breve corcho, pero bien labrado,\\
un rubio hijo de una encina hueca,\\
dulcísimo panal, a cuya cera\\
su néctar vinculó la primavera.\par\pend

\begin{center}
	[XXVII]
\end{center}\pstart
Caluroso, al ar[r]oyo da las manos,\\
y con ellas las ondas a su frente,\\
entre dos mirthos que, de espuma canos,\\
dos verdes garças son de la corriente.\\
Vagas cortinas de volantes vanos\\
corrió Fabonio lisongeramente\\
a la (de viento quando no sea) cama\\
de frescas sombras, de menuda grama.\par\pend
\newpage

\begin{center}
	[XXVIII]
\end{center}\pstart
La nimpha, pues, la sonorosa plata\\
bullir sintió del arroyuelo apenas,\\
quando, a los verdes márgenes ingrata,\\
\edtext{segur}{\Afootnote{seguir \textit{Ch}, \textit{G1}}}\footnoteE{La lección \textit{segur} la recoge Pellicer en su \textit{Lecciones solemnes}; \textcite{Alonso1967} la defiende por considerarla \textit{lectio difficilior}. No hay manera de determinar si alguna es un error; ambas lecciones podrían ser correctas.} se hizo de sus azucenas.\\
Huyera; mas tan frío se desata\\
un temor perezoso por sus venas,\\
que a la precisa fuga, al presto vuelo,\\
grillos de nieve fue, plumas de hielo.\par\pend

\begin{center}
	[XXIX]
\end{center}\pstart
Fruta en mimbres halló, leche exprimida\\
en juncos, miel en corcho, mas sin dueño;\\
si bien al dueño debe, agradecida\edtext{}{\Afootnote[nosep]{agredecida \textit{err. G1}}},\\
su deidad culta, venerado el sueño.\\
A la ausencia mil veces ofrecida,\\
este de cortesía no pequeño\\
indicio la dexó --aunque \edtext{estatua}{\Afootnote{estaua \textit{G1}}} helada--\\
más discursiva y menos alterada.\par\pend

\begin{center}
	[XXX]
\end{center}\pstart
No al cíclope atribuye, no, la ofrenda:\\
no a sátiro lascivo, ni a otro feo\\
morador de las selvas; cuya rienda\\
el sueño aflija, que afloxó el deseo.\\
El niño dios, entonces, de la venda,\\
ostentación gloriosa, alto tropheo\\
quiere que al árbol de su madre sea\\
el desdén hasta allí de Galathea.\par\pend

\begin{center}
	[XXXI]
\end{center}\pstart
Entre las ramas del que más se lava\\
en el arroyo, mirtho levantado,\\
carcax de cristal hizo, sino aljava\\
su blanco pecho, de un harpón\footnoteE{\textit{h} = hipercorrección.} dorado.\\
El monstro de rigor, la fiera braba\\
mira la ofrenda ya con más cuidado,\\
y aun siente que a su dueño sea, devoto,\\
confuso alcaide más, el verde soto.\par\pend

\begin{center}
	[XXXII]
\end{center}\pstart
Llamáralo, aunque muda, mas no sabe\\
el nombre articular, que más querría;\\
ni lo ha visto, si bien pincel süave\\
le ha vosquexado ya en su fantasía.\\
Al pie -no tanto ya, del temor, grave-\\
fía su intento; y, tímida en la umbría\\
cama de campo y campo de batalla,\\
fingiendo sueño al canto garzón halla.\par\pend

\begin{center}
	[XXXIII]
\end{center}\pstart
El vulto vio, y, haciéndolo dormido,\\
librada en un pie toda sobre él pende\\
(urbana al sueño, bárbara al mentido\\
rhetórico silencio que no entiende):\\
no el ave reina, así, el fragoso nido\\
corona immóbil, \edtext{mientras}{\Afootnote{quando no \textit{G1}}} no desciende\\
--rayo con plumas-- al milano pollo\\
que la eminencia abriga de un escollo,\par\pend

\begin{center}
	[XXXIV]
\end{center}\pstart
como la nimpha bella, compitiendo\\
con el garçón dormido en cortesía,\\
no sólo para más el dulce estruendo\\
del lento arroyo emmudecer querría.\\
A pesar luego de las \edtext{ramas}{\Afootnote{armas \textit{G1}}}, viendo\\ colorido el bosquexo que ya avía\\
en su imaginación Cupido hecho\\
con el pincel que le clavó su pecho,\par\pend

\begin{center}
	[XXXV]
\end{center}\pstart
de sitio mejorada, atenta mira\\
en la disposición robusta, aquello\\
que, si por la süave no la admira,\\
es fuerça que la admire por lo bello.\\
Del casi tramontado sol aspira\\
a los confusos rayos, su cabello;\\
flores su bozo es, cuyas colores,\\
como duerme la luz, niegan las flores.\par\pend
\relax
\vfill
\newpage

\begin{center}
	[XXXVI]
\end{center}\pstart
En la rústica greña yace oculto\\
el áspid, del intonso prado ameno,\\
antes que del peinado jardín culto\\
en el lascivo, regalado\edtext{}{\Afootnote[nosep]{regado \textit{err. G1}}} seno:\\
en lo viril desata de su vulto\\
lo más dulce el Amor, de su veneno;\\
bébelo Galathea, y da otro paso\\
por apurarle la ponzoña al vaso.\par\pend

\begin{center}
	[XXXVII]
\end{center}\pstart
Acis --aún más de aquello \edtext{que piensa}{\Afootnote{que dispensa \textit{G1}}}\\
la brúxula del sueño vigilante--,\\
alterada la nimpha esté o suspensa,\\
Argos es siempre atento a su semblante,\\
lince penetrador de lo que piensa,\\
cíñalo bronce, o \edtext{múrelo}{\Afootnote{muerelo \textit{Ch}}} diamante:\\
que en sus palladïones \edtext{Amor}{\Afootnote{amo \textit{G1}}} ciego,\\
sin romper muros, introduce fuego.\par\pend 

\begin{center}
	[XXXVIII]
\end{center}\pstart
El sueño de sus miembros sacudido,\\
gallardo el joven la persona ostenta,\\
y al marfil luego de sus pies rendido,\\
el cothurno besar dorado intenta.\\
Menos ofende el rayo prevenido,\\
al marinero, menos la tormenta\\
prevista le turbó o prognosticada:\\
Galathea lo diga, salteada.\par\pend 

\begin{center}
	[XXXIX]
\end{center}\pstart
Más agradable y menos zahareña,\\
al mancebo levanta venturoso,\\
dulce ya concediéndole y risueña,\\
paces no al sueño, treguas sí al reposo.\\
Lo cóncavo hacía de una peña\\
a un fresco sitïal dosel umbroso,\\
y verdes celosías unas hiedras,\\
trepando troncos y abrazando piedras.\par\pend
\relax
\vfill
\newpage

\begin{center}
	[XL]
\end{center}\pstart
Sobre una alfombra, que imitara en vano\\
el tirio sus matices (si bien \edtext{era}{\Afootnote{ora \textit{Ch}}}\\
de quantas sedas la hiló, gusano,\\
y, artífice, texió la Primavera)\\
reclinados, al mirtho mas lozano,\\
una y otra lasciva, si ligera,\\
paloma se \edtext{caló}{\Afootnote{callo \textit{G1}}}, cuyos gemidos\\
--trompas de Amor-- alteran sus oídos.\par\pend

\begin{center}
	[XLI]
\end{center}\pstart
El ronco arrullo al joven solicita;\\
mas, con desvíos Galathea suaves,\\
a su audacia los términos limita,\\
y el aplauso al concento de las aves.\\
Entre las ondas y la fruta, imita\\
Acis al siempre ayuno en penas graves:\\
que, en tanta gloria, infierno son no breve,\\
fugitivo cristal, pomos de nieve.\par\pend

\begin{center}
	[XLII]
\end{center}\pstart
No a las palomas concedió Cupido\\
juntar de sus dos picos los rubíes,\\
quando al clavel el joven atrevido\\
las dos hojas le chupa carmesíes.\\
Quantas produce Papho, engendra Gnido,\\
negras vïolas, blancas alhelíes,\\
llueven sobre el que Amor quiere que sea\\
tálamo de Acis ya y de Galathea.\par\pend

\begin{center}
	[XLIII]
\end{center}\pstart
Su aliento humo, sus relinchos fuego,\\
si bien su freno espumas, illustrava\\
las columnas \edtext{Ethón}{\Afootnote{Phaeton \textit{G1}}} que erigió el griego,\\
do el carro de la luz sus ruedas lava,\\ quando, de amor el fiero jayán ciego,\\
la cerviz le oprimió a una roca brava,\\
que a la playa, de escollos no desnuda,\\
linterna es ciega y atalaya muda.\par\pend
\relax
\vfill
\newpage

\begin{center}
	[XLIV]
\end{center}\pstart
\edtext{Arbitro}{\Afootnote{Arbitrios \textit{G1}}} de montañas y ribera,\\
aliento dio, en la cumbre de la roca,\\
\edtext{a los albogues que agregó la cera}{\Afootnote{aquantas canas agregó la çera \textit{G1}}},\\
el prodigioso fuelle de su voca;\\
la nimpha los oyó, y ser más quisiera\\
breve flor, hierba humilde, y tierra poca,\\
que de su nuevo tronco vid lasciva,\\
muerta de amor, y de temor no viva.\par\pend

\begin{center}
	[XLV]
\end{center}\pstart
Mas --cristalinos pámpanos sus braços--\\
amor la implica, si el temor la anuda,\\
al infelice olmo que pedazos\\
la segur de los zelos hará aguda.\\
Las cavernas en tanto, los ribazos\\
que ha prevenido la zampoña ruda,\\
el trueno de la voz fulminó luego:\\
¡referidlo, Pïérides, os ruego!\par\pend

\begin{center}
	[XLVI]
\end{center}\pstart
«¡Oh bella Galathea, mas süave\\
que los claveles que tronchó la aurora;\\
blanca más que las plumas de aquel ave\\
que dulce muere y en las aguas mora;\\
igual en pompa al páxaro que, grave,\\
su manto azul de tantos ojos dora\\
quantas el celestial zaphiro estrellas!\\
¡Oh tú, que en dos incluyes las más bellas!\par\pend

\begin{center}
	[XLVII]
\end{center}\pstart
»Dexa las ondas, dexa el rubio choro\\
de las hijas de Tetis, y el mar vea,\\
quando niega la luz un carro de oro,\\
que en dos la restituye Galathea.\\
Pisa la arena, que en la arena adoro\\
quantas el blanco pie conchas platea,\\
cuyo bello contacto puede hacerlas\\
sin concebir roció, parir perlas.\par\pend
\relax
\vfill
\newpage

\begin{center}
	[XLVIII]
\end{center}\pstart
»Sorda hija del mar, cuyas orejas\\
a mis gemidos son rocas al viento:\\
o dormida te hurten a mis quexas\\
purpúreos troncos de corales ciento,\\
o \edtext{al disonante}{\Afootnote{altisonante \textit{G1}}} número de almejas\\
--marino, si agradable no, instrumento--\\
choros texiendo estés, escucha un día\\
mi voz, por dulce, quando no por mía.\par\pend

\begin{center}
	[XLX]
\end{center}\pstart
»Pastor soy, mas tan rico de ganados,\\
que los valles impido más vacíos,\\
los cerros desparezco levantados,\\
y los caudales seco de los ríos;\\
no los que, de sus ubres desatados,\\
\edtext{o deribados}{\Afootnote{y despeñados \textit{G1}}} de los ojos míos,\\
leche corren y lágrimas; que iguales\\
en número a mis bienes son mis males.\par\pend

\begin{center}
	[L]
\end{center}\pstart
»Sudando néctar, lambicando olores,\\
senos que ignora aun la golosa cabra,\\
corchos me guardan, más que aveja flores\\
liba inquïeta, ingenïosa labra;\\
troncos \edtext{me ofrecen}{\Afootnote{me dan \textit{G1}}} árboles mayores,\\
cuyos enxambres, o el abril los abra,\\
o los desate el mayo, ámbar distilan\\
y en ruecas de oros rayos de sol hilan.\par\pend

\begin{center}
	[LI]
\end{center}\pstart
»Del Júpiter soy hijo, de las ondas,\\
aunque pastor; si tu desdén no espera\\
\edtext{a que el monarcha}{\Afootnote{al gran monarcha \textit{G1}}} de esas grutas hondas,\\
en \edtext{trono}{\Afootnote{thono \textit{Ch}; tronco \textit{G1}}} de cristal te abrace nuera,\\
Poliphemo te llama, no te escondas;\\
que tanto esposo admira la ribera,\\
qual otro no vio Phebo, más robusto,\\
del perezoso Bolga al Indo adusto.\par\pend
\relax
\vfill
\newpage

\begin{center}
	[LII]
\end{center}\pstart
»Sentado, a la alta palma no perdona\\
su dulce fruto mi robusta mano;\\
en pie, sombra capaz es mi persona\\
de innumerables cabras el verano.\\
¿Qué mucho, si de \edtext{nubes}{\Afootnote{nieues \textit{G1}}} se corona\\
por igualarme la montaña en vano,\\
y en los cielos, desde esta roca, puedo\\
escribir mis desdichas con el dedo?\par\pend

\begin{center}
	[LIII]
\end{center}\pstart
»Marítimo alcïón, roca eminente\\
sobre sus huebos coronaba, el día\\
que espejo de zaphiro fue luciente\\
la playa azul, de la persona mía.\\
Miréme, y lucir vi un sol en mi frente,\\
quando en el cielo un ojo se veía:\\
neutra el agua dudaba a quál fe preste,\\
o al cielo humano, o al cíclope celeste.\par\pend

\begin{center}
	[LIV]
\end{center}\pstart
»Registra en otras puertas al venado\\
sus años, su cabeza colmilluda\\
la fiera, cuyo cerro levantado,\\
de helvecias picas es muralla aguda;\\
la humana suya el caminante errado\\
dio ya a mi cueva, de piedad desnuda,\\
alvergue hoy, por tu causa, al peregrino,\\
do halló reparo, si perdió camino.\par\pend

\begin{center}
	[LV]
\end{center}\pstart
»En tablas dividida, rica nave\\
besó la playa miserablemente,\\
de quantas vomitó riqueças grave\\
por las vocas del Nilo de Orïente.\\
Yugo aquel día, y yugo bien süave,\\
del fiero mar a la sañuda frente\\
imponiéndole estava (si no al viento\\
dulcísimas coyundas) mi instrumento,\par\pend
\relax
\vfill
\newpage

\begin{center}
	[LVI]
\end{center}\pstart
»quando entre globos de agua, entregar veo\\
a las arenas ligurina haya,\\
en caxas los aromas del Sabeo,\\
en cofres las riquezas de Cambaya:\\
delicias de aquel mundo, ya tropheo\\
de Scila, que, ostentado en nuestra playa,\\
lastimoso despojo fue dos días\\
a las que esta montaña engendra harpías.\par\pend

\begin{center}
	[LVII]
\end{center}\pstart
»Segunda tabla a un ginovés mi gruta\\
de su persona fue, de su hazienda;\\
la una reparada, la otra enjuta,\\
relación del naufragio hiço horrenda.\\
Luciente paga de la mejor fruta\\
que en yerbas se recline, o en hilos penda,\\
colmillo fue del animal que el Ganges\\
sufrir muros le vio, romper phalanges:\par\pend

\begin{center}
	[LVIII]
\end{center}\pstart
»arco digno, gentil, bruñida aljava,\\
obras ambas de artífice prolixo,\\
y de Malaco rey a deidad Java\edtext{}{\Afootnote[nosep]{laua \textit{err. Ch}}}\\
alto don, según ya mi huésped dixo.\\
De aquel la mano, de ésta el hombro agrava;\\
convencida la madre, imita al hijo:\\
serás a un tiempo en estos horizontes\\
Venus del mar, Cupido de los montes.»\par\pend

\begin{center}
	[LIX]
\end{center}\pstart
Su horrenda voz, no su dolor interno,\\
cabras aquí le interrumpieron, quantas\\
--vagas el pie, sacrílegas el cuerno--\\
a Baccho se atrevieron en sus plantas.\\
Mas, conculcado el pámpano más tierno\\
viendo el fiero pastor, vozes él tantas,\\
y tantas despidió la honda piedras,\\
que el muro penetraron de las yedras.\par\pend
\relax
\vfill
\newpage

\begin{center}
	[LX]
\end{center}\pstart
D\textit{e} los nudos, con esto\edtext{}{\Afootnote[nosep]{q\textit{ue} honestos \textit{err. Ch}}}, más süaves.\\
los dulces dos amantes desatados,\\
por duras guijas, por espinas graves\\
solicitan el mar con pies alados:\\
tal, redimiendo de importunas aves\\
incauto meseguero sus sembrados,\\
de liebres dirimió copia, así, amiga,\\
que vario sexo unió y un surco abriga.\par\pend

\begin{center}
	[LXI]
\end{center}\pstart
Viendo el fiero jayán, con paso mudo\\
correr al mar la fugitiva nieve\\
(que a tanta vista el líbico desnudo\\
registra el campo de su adarga breve)\\
y al garçón viendo, quantas mover pudo\edtext{}{\Afootnote[nosep]{puedo \textit{err. G1}}}\\
zeloso trueno, antiguas hayas mueve:\\
tal, antes que la opaca nube rompa,\\
previene rayo fulminante trompa.\par\pend

\begin{center}
	[LXII]
\end{center}\pstart
Con vïolencia desgajó infinita,\\
mayor punta de la excelsa roca,\\
que al joven, sobre quien la precipita,\\
urna es mucha, pirámide no poca.\\
Con lágrimas la nimpha solicita\\
las deidades del mar, que Acis invoca:\\
concurren todas, y el peñasco duro\\
la sangre que exprimió, cristal fue puro.\par\pend

\begin{center}
	[LXIII]
\end{center}\pstart
Sus miembros lastimosamente opresos\\
del escollo fatal fueron apenas,\\
que los pies de los árboles más gruesos\\
calçó el líquido aljófar de sus venas.\\
\edtext{Corriente}{\Afootnote{luçiente \textit{G1}}} plata al fin sus blancos huesos,\\
lamiendo flores y argentando arenas,\\
a Doris llega, que, con llanto pío,\\
yerno lo saludó, lo aclamó río.\par\pend
\endnumbering
\relax
\vfill
\end{document}
